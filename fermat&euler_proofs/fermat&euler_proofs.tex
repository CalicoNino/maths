\documentclass{article}
\usepackage[utf8]{inputenc}
\usepackage[english]{babel}

\usepackage{amsthm}
\usepackage{amsmath}
\usepackage[makeroom]{cancel}
\usepackage{amssymb}
\usepackage{enumitem}

\title{Fermat's Little Theorem \& Euler's Theorem Proofs}
\author{Karim El Shenawy}
\date{February 2021}

\usepackage{natbib}
\usepackage{graphicx}

\begin{document}

\maketitle

\section*{Introduction}
This course notebook is the collection of theorem proofs, exercises and answers from Unit 4 of the Number Theory Through Inquiry (Mathematical Association of America Textbooks).

\section*{Theorems to Mark}

\subsection*{4.3 Theorem} 
\quad \textit{Let a, b and n be natural numbers with $(a, n) = 1$. If $a \equiv b \;(\bmod\; n)$, then $(b, n) = 1$.}

\begin{proof}
Given a, b and n as natural numbers with $(a, n) = 1$. Then by Theorem 1.39, $(a, n) = 1 \Longrightarrow 1 = ax + ny, \exists x, y \in \mathbf{Z}$. Let's assume that $a \equiv b \;(\bmod\; n)$ which also implies that n divides $a - b$ since $a \equiv b \;(\bmod\; n) \Longrightarrow a - b \equiv 0 \;(\bmod\; n) \Longrightarrow n \mid a - b$. By definition, $n \mid a - b$ is also $ a - b = nd, \exists d \in \mathbf{Z}$.

Now by direct proof and considering that $1 = ax + ny$ and $a= nd + b$;
\begin{align*}
    && 1 &= ax + ny&&\\
    && 1 &= (nd + b)x + ny&& \text{By substituting $a= nd + b$}\\
    && 1 &= ndx + bx + ny&&\\
    && 1 &= n(dx + y) + bx&&\\
    && 1 &= nz + bx&& \text{Since $d,x,y \in \mathbf{Z}$, assume that some integer $z = dx + y$}\\
    && 1 &= gcd(b, n)&& \text{By Theorem 1.39}
\end{align*}
Thus If $(a,n) = 1$ and $a \equiv b \;(\bmod\; n)$, then $(b, n) = 1$.
\end{proof}

\subsection*{4.10 Theorem} 
\quad \textit{Let a and n be natural numbers with $(a, n) = 1$, let $k = ord_n(a)$, and let m be a natural number. Then $a^m \equiv 1 \;(\bmod\; n)$ if and only if $k \mid m$.}

\begin{proof}
Let a, n and m $\in \mathbf{N}$ be natural numbers with $(a, n) = 1$, let $k = ord_n(a)$. Suppose $a^m \equiv 1 \;(\bmod\; n)$, by the Division Algorithm, we can express $m = qk + r, 0 \leq r < k$. Then, 
\begin{align*}
    && a^m &\equiv a^{qk+r} \;(\bmod\; n)&&\\
    && a^m &\equiv a^{qk} \cdot a^r \;(\bmod\; n)&&\\
    && a^m &\equiv 1^q \cdot a^r \;(\bmod\; n)&&\\
    && a^m &\equiv a^r \;(\bmod\; n)&&\\
    && a^m &\equiv 1 \;(\bmod\; n)&& \text{Since $a^m \equiv 1 \;(\bmod\; n)$}
\end{align*}
The above implies that r must be 0, which also implies that $m = kq$. Thus, $k \mid m$.
Conversely, lets suppose that $k \mid m$, then,
\begin{align*}
    && a^m &\equiv a^{qk} \;(\bmod\; n)&&\\
    && a^m &\equiv 1^q \;(\bmod\; n)&&\\
    && a^m &\equiv 1 \;(\bmod\; n)&&
\end{align*}
Thus, $a^m \equiv 1 \;(\bmod\; n)$ if and only if $k \mid m$.
\end{proof}

\subsection*{4.15 Theorem (Fermat's Little Theorem, Version I)} 
\quad \textit{If p is a prime and a is an integer relatively prime to p, then $a^{(p-1)} \equiv 1 \;(\bmod\; p)$.}

\begin{proof}
Given a prime p, let the set $\{1,2,3,...,p-1\}$ be all the number modulo p, by Theorem 4.14. We can claim that some integer a is in $\{1,2,3,...,p-1\}$ since $(p, a) = 1$.\\
If we multiply the numbers in the set $\{1,2,3,...,p-1\}$ by a, we obtain $\{a,2a,3a,...,(p-1)a\}$ and we know by Theorem 4.14 that;
\begin{align*}
    &&a \cdot 2a \cdot 3a \cdot \cdot \cdot \cdot (p-1)a &\equiv 1 \cdot 2 \cdot 3 \cdot \cdot \cdot \cdot (p-1) \;(\bmod\; p)&&\\
    &&a^{p-1} \cdot (p-1)! &\equiv (p-1)! \;(\bmod\; p)&&\\
    &&a^{p-1} &\equiv 1 \;(\bmod\; p)&& (p-1)! \;(\bmod\; p) \equiv 1
\end{align*}
Since we know that all values in $\{1,2,3,...,p-1\}$ are all modulo p, thus $(p-1)! \;(\bmod\; p) \equiv 1$. Hence, $a^{(p-1)} \equiv 1 \;(\bmod\; p)$.holds.
\end{proof}

\subsection*{4.25 Lemma} 
\quad \textit{If p is prime and i is a natural number less than p, then p divides $\binom{p}{i}$.}

\begin{proof}
Given p prime and i, a natural number and less than p, we can expand $\binom{p}{i}$;
\begin{align*}
    &&\binom{p}{i} &= \frac{p!}{i!(p-i)!}&&\\
    &&\binom{p}{i} &= \frac{p(p-1)(p-2)\cdot\cdot\cdot(p-(i-1))(p-i)!}{i!(p-i)!}&&\\
    &&\binom{p}{i} &= \frac{p(p-1)(p-2)\cdot\cdot\cdot(p-(i-1))}{i!}&&\\
    &&\binom{p}{i}\cdot i! &= p(p-1)(p-2)\cdot\cdot\cdot(p-(i-1))&&
\end{align*}
Now, p divides p thus we can express it as the following;
\begin{center}
    $p \mid p \Longrightarrow p \mid p(p-1)(p-2)\cdot\cdot\cdot(p-(i-1))$
    $\Longrightarrow p \mid \binom{p}{i}\cdot i!$
    $\Longrightarrow p \mid \binom{p}{i}\cdot i(i-1)(i-2)\cdot\cdot\cdot3\cdot2\cdot1$
\end{center}
Since i is less than p, p can not divide $i!$. Thus we end up with $p \mid \binom{p}{i}$. Therefore, p must divide $\binom{p}{i}$.
\end{proof}

\subsection*{4.38 Theorem} 
\quad \textit{Let p be a prime and let a and b be integers such that $1 < a, b < p-1$ and $ab \equiv 1 \;(\bmod\; p)$. Then $a \neq b$.}

\begin{proof}
Let p be a prime and let a and b be integers such that $1 < a, b < p-1$ and $ab \equiv 1 \;(\bmod\; p)$. Now lets assume, by contradiction, that $a = b$. Then, $ab \equiv a^2 \equiv 1 \;(\bmod\; p)$. Also, by definition, $a^2 \equiv 1 \;(\bmod\; p) \Longrightarrow p \mid a^2 -1$ where $a^2-1 = (a+1)(a-1)$. Since p is a prime, can divide $a-1$ or $a+1$.\\
However, since we know that $1 < a < p-1$, which can also be expressed as $1 < a + 1< p$, this implies that p can not divide $a$. Similarly, since $a < p-1$, p can not divide $a+1$ either. This signifies that $p \nmid a^2 = ab, a = b$. Thus $a \neq b$. 
\end{proof}

\subsection*{4.41 Theorem (Wilson's Theorem)} 
\quad \textit{If p is a prime, then $(p-1)! \equiv -1 \;(\bmod\; p)$.}

\begin{proof}
By Theorem, 4.40, suppose p is a prime larger than 2, then $2 \cdot 3 \cdot 4 \cdot\cdot\cdot\cdot(p-2) \equiv 1 \;(\bmod\; p)$. Now, let's multiply (p-1) in both sides;
\begin{align*}
    &&2 \cdot 3 \cdot 4 \cdot\cdot\cdot\cdot(p-2) &\equiv 1 \;(\bmod\; p) &&\\
    &&2 \cdot 3 \cdot 4 \cdot\cdot\cdot\cdot(p-2)\cdot (p-1) &\equiv (p-1) \;(\bmod\; p) &&\\
    &&2 \cdot 3 \cdot 4 \cdot\cdot\cdot\cdot(p-2)\cdot (p-1) &\equiv -1 \;(\bmod\; p) &&\\
    &&(p-1)! &\equiv -1 \;(\bmod\; p) &&
\end{align*}
Thus, if p is a prime, then $(p-1)! \equiv -1 \;(\bmod\; p)$.
\end{proof}

\section*{Practice Theorems from A Modular World}

\subsection*{4.1 Exercise} 
\quad \textit{For $i = 0,1,2,3,4,5$ and $6$, find the number in the canonical complete residue system to which $2^i$ is congruent modulo 7. In other words, compute $2^0 \;(\bmod\; 7), 2^1 \;(\bmod\; 7), 2^2 \;(\bmod\; 7),...,2^6 \;(\bmod\; 7)$.}

\begin{itemize}
    \item $2^0 \equiv 0 \;(\bmod\; 7)$
    \item $2^1 \equiv 2 \;(\bmod\; 7)$
    \item $2^2 \equiv 4  \;(\bmod\; 7)$
    \item $2^3 \equiv 8  \;(\bmod\; 7)$
    \item $2^4 \equiv 16 \equiv 5 \;(\bmod\; 7)$
    \item $2^5 \equiv 32 \equiv 4 \;(\bmod\; 7)$
    \item $2^6 \equiv 64 \equiv 1 \;(\bmod\; 7)$
\end{itemize}

\subsection*{4.2 Theorem} 
\quad \textit{Let a and n be natural numbers with $(a, n) = 1$. Then $(a^j, n) = 1$ for any natural number j.}

\begin{proof}
Given that natural numbers a and n have gcd(a, n) = 1 implies that a and n are relatively prime. Therefore, $a^j$ must also have no primes factors in common with n.
\end{proof}

\subsection*{4.3 Theorem} 
\quad \textit{Let a, b and n be natural numbers with $(a, n) = 1$. If $a \equiv b \;(\bmod\; n)$, then $(b, n) = 1$.}

\begin{proof}
Given a, b and n as natural numbers with $(a, n) = 1$. Then by Theorem 1.39,$(a, n) = 1 \Longrightarrow 1 = ax + ny, \exists x, y \in \mathbf{Z}$. Let's assume that $a \equiv b \;(\bmod\; n)$ which also implies that n divides $a - b$ since $a \equiv b \;(\bmod\; n) \Longrightarrow a - b \equiv 0 \;(\bmod\; n) \Longrightarrow n \mid a - b$. By definition, $n \mid a - b$ is also $ a - b = nd, \exists d \in \mathbf{Z}$.

Now by direct proof and considering that $1 = ax + ny$ and $a= nd + b$;
\begin{align*}
    && 1 &= ax + ny&&\\
    && 1 &= (nd + b)x + ny&& \text{By substituting $a= nd + b$}\\
    && 1 &= ndx + bx + ny&&\\
    && 1 &= n(dx + y) + bx&&\\
    && 1 &= nz + bx&& \text{Since $d,x,y \in \mathbf{Z}$, assume that some integer $z = dx + y$}\\
    && 1 &= gcd(b, n)&& \text{By Theorem 1.39}
\end{align*}
Thus If $(a,n) = 1$ and $a \equiv b \;(\bmod\; n)$, then $(b, n) = 1$.
\end{proof}

\subsection*{4.4 Theorem} 
\quad \textit{Let a and n be natural numbers. Then there exist natural numbers i and j, with $i \neq j$, such that $a^i \equiv a^j \;(\bmod\; n)$.}

\begin{proof}
Suppose natural number a, n, m, i and j exist such that $a^m \equiv 1 \;(\bmod\; n)$ and $i \neq j$. Now by direct proof;
\begin{align*}
    && a^m \equiv 1 \;(\bmod\; n)&&\\
    && a^m(a^{10}) \equiv a^{10} \;(\bmod\; n)&& \text{By multiplying $a^{10}$ on both sides}\\
    && a^{m-10} \equiv a^{10} \;(\bmod\; n)&& \text{Exponent Properties}
\end{align*}
Therefore there must exist an i and j with $i \neq j$ such that $a^i \equiv a^j \;(\bmod\; n)$ since $m-10 \neq 10$.
\end{proof}

\subsection*{4.5 Theorem} 
\quad \textit{Let a, b, c and n be integers with $n > 0$. If $ac \equiv bc \;(\bmod\; n)$ and $(c, n) = 1$, then $a \equiv b \;(\bmod\; n)$.}

\begin{proof}
Given integers a, b, c and n with $n > 0$. Then $ac \equiv bc \;(\bmod\; n) \Longrightarrow n \mid c(a-b)$ by definition. However, assuming that c and n are relatively prime by $(c, n) = 1$, n only divides $a - b$, $n \mid a - b$. Therefore, by definition, $n \mid a - b \Longrightarrow a - b \equiv 0 \;(\bmod\; n) \Longrightarrow a \equiv b \;(\bmod\; n)$.
\end{proof}

\subsection*{4.6 Theorem} 
\quad \textit{Let a and n be natural numbers with $(a, n) = 1$. Then there exists a natural number k such that $a^k \equiv 1 \;(\bmod\; n)$.}

\begin{proof}
Given natural numbers a and n with $(a, n) = 1$. Then by Theorem 4.2, $(a^k, n) = 1, \exists k \in \mathbf{N}$. Also by Theorem 4.4, $a^i \equiv a^j \;(\bmod\; n), \exists i, j \in \mathbf{N}$ with $i \neq j$. Let $k = i - j$, thus $a^i - a^j \equiv a^k \equiv 0 \;(\bmod\; n)$. However, since $(a^k, n) = 1$, $a^{k} \not\equiv 0 \;(\bmod\; n)$ and it can only be $a^{k} \equiv 1 \;(\bmod\; n)$.
\end{proof}

\subsection*{4.7 Question} 
\quad \textit{Choose some relatively prime natural numbers a and n and compute the order of a modulo n. Frame a conjecture concerning how large the order of a modulo n can be, depending on n.}

\begin{itemize}
    \item $(a, n) = (5, 12) = 1$\\
          $5^2 \equiv 1 \;(\bmod\; 15)$, order = 2
    \item $(a, n) = (2, 3) = 1$\\
          $2^2 \equiv 1 \;(\bmod\; 3)$, order = 2
    \item $(a, n) = (7, 10) = 1$\\
          $7^4 \equiv 1 \;(\bmod\; 10)$, order = 4
\end{itemize}

\textbf{Conjecture.} \textit{The order of a modulo n  will always be under n.}

\subsection*{4.8 Theorem} 
\quad \textit{Let a and n be natural numbers with $(a, n) = 1$ and let $k = ord_n(a)$. Then the numbers $a^1, a^2,...,a^k$ are pairwise incongruent modulo n.}

\begin{proof}
Given natural numbers a and n with $(a, n) = 1$ and let $k = ord_n(a)$, lets suppose that there exists integers i and j such that $a^i \equiv a^j \;(\bmod\; n)$ and that $1 \leq j < i < k$. By Theorem 4.2, we know that $a^i \equiv a^j \;(\bmod\; n) \Longrightarrow a^{i-j} \equiv 1 \;(\bmod\; n)$ with $i-j < k$ since $(a, n) = 1$. However, k is considered to be the smallest natural number where $a^k \equiv 1 \;(\bmod\; n)$ by $k = ord_n(a)$. Thus, if $(a, n) = 1$ and $k = ord_n(a)$, then the numbers $a^1, a^2,...,a^k$ are pairwise incongruent modulo n, i.e. values $\;(\bmod\; n)$ never repeat.
\end{proof}

\subsection*{4.9 Theorem} 
\quad \textit{Let a and n be natural numbers with $(a, n) = 1$ and let $k = ord_n(a)$. For any natural number m, $a^m$ is congruent modulo n to one of the numbers $a^1, a^2,...,a^k$.}

\begin{proof}
Given the natural numbers a and n with $(a, n) = 1$ and let $k = ord_n(a)$, we know  that $a^k \equiv 1 \;(\bmod\; n)$. Suppose m exists and is any natural number with $1 < k < m$. By the Division Algorithm, we can express $m = qk + r, 0 \leq r < k$. Now, $a^m \equiv a^{qk+r} \equiv a^{k}^{q}\cdot a^r \equiv 1^q \cdot a^r \;(\bmod\; n)$. Thus $a^m$ is congruent modulo n to one of the numbers $a^1, a^2,...,a^k$.
\end{proof}

\subsection*{4.10 Theorem} 
\quad \textit{Let a and n be natural numbers with $(a, n) = 1$, let $k = ord_n(a)$, and let m be a natural number. Then $a^m \equiv 1 \;(\bmod\; n)$ if and only if $k \mid m$.}

\begin{proof}
Let a, n and m $\in \mathbf{N}$ be natural numbers with $(a, n) = 1$, let $k = ord_n(a)$. Suppose $a^m \equiv 1 \;(\bmod\; n)$, by the Division Algorithm, we can express $m = qk + r, 0 \leq r < k$. Then, 
\begin{align*}
    && a^m &\equiv a^{qk+r} \;(\bmod\; n)&&\\
    && a^m &\equiv a^{qk} \cdot a^r \;(\bmod\; n)&&\\
    && a^m &\equiv 1^q \cdot a^r \;(\bmod\; n)&&\\
    && a^m &\equiv a^r \;(\bmod\; n)&&\\
    && a^m &\equiv 1 \;(\bmod\; n)&& \text{Since $a^m \equiv 1 \;(\bmod\; n)$}
\end{align*}
The above implies that r must be 0, which also implies that $m = kq$. Thus, $k \mid m$.
Conversely, lets suppose that $k \mid m$, then,
\begin{align*}
    && a^m &\equiv a^{qk} \;(\bmod\; n)&&\\
    && a^m &\equiv 1^q \;(\bmod\; n)&&\\
    && a^m &\equiv 1 \;(\bmod\; n)&&
\end{align*}
Thus, $a^m \equiv 1 \;(\bmod\; n)$ if and only if $k \mid m$.
\end{proof}

\subsection*{4.11 Theorem} 
\quad \textit{Let a and n be natural numbers with $(a, n) = 1$. Then $ord_n(a) < n$.}

\begin{proof}
Let a and n $\in \mathbf{N}$ with $(a, n) = 1$ and suppose that that $ord_n(a) \mid k$. Then $k = ord_n(a) \cdot d, \exists d \in \mathbf{Z}$. Now,
\begin{align*}
    && a^k &\equiv a^{ord_n(a) \cdot d} \;(\bmod\; n)&&\\
    && a^k &\equiv 1^d \;(\bmod\; n)&& \text{By Definition}\\
    && a^k &\equiv 1 \;(\bmod\; n)&&
\end{align*}
Conversely, if $a^k \equiv 1 \;(\bmod\; n)$, we can show that $ord_n(a) \mid k$ by the Division Algorithm with $k = ord_n(a) \cdot d + r, \exists d, r \in \mathbf{Z}$ where $0 \leq r < ord_n(a)$. Then, 
\begin{align*}
    && a^k &\equiv a^{ord_n(a) \cdot d + r} \;(\bmod\; n)&&\\
    && a^k &\equiv a^{ord_n(a) \cdot d} \cdot a^r \;(\bmod\; n)&&\\
    && a^k &\equiv 1^d \cdot a^r \;(\bmod\; n)&& \text{By Definition}\\
    && a^k &\equiv a^r \;(\bmod\; n)&&
\end{align*}
Since we know that $0 \leq r < ord_n(a)$ and $a^k \equiv 1 \;(\bmod\; n)$, r must be 0. Thus, $ord_n(a) \mid k$.

Now, if $ord_n(a) \mid k$ then $ord_n(a) \leq k$ where k is the smallest natural number $k < n$. Thus, $ord_n(a) \leq k < n$ and $ord_n(a) < n$.
\end{proof}

\subsection*{4.12 Exercise} 
\quad \textit{Compute $a^{p-1} \;(\bmod\; p)$ for various numbers a and primes p, and make a conjecture.}

\begin{itemize}
    \item $3^{2-1} \equiv 3 \;(\bmod\; 2)$
    \item $4^{2-1} \equiv 4 \;(\bmod\; 2)$
    \item $4^{3-1} \equiv 4^2 \;(\bmod\; 2)$
    \item $4^{5-1} \equiv 4^4 \;(\bmod\; 2)$
    \item $4^{7-1} \equiv 4^6 \;(\bmod\; 2)$
    \item $4^{11-1} \equiv 4^{10} \;(\bmod\; 2)$
\end{itemize}

\textbf{Conjecture.} \textit{If a is a prime different from p, then $a^{p-1} \equiv 1 \;(\bmod\; p)$.}

\subsection*{4.13 Theorem} 
\quad \textit{Let p be a prime and let a be an integer not divisible by p; that is, $(a, p) = 1$. Then $\{a,2a,3a,...,pa\}$ is a complete residue system modulo p.}

\begin{proof}
Let p be a prime where $(p, a) = 1$ with $a \in \mathbf{Z}$ and also consider the set $P = \{1,2,3,...,p\}$. For elements $p_1$ and $p_2 \in P$, suppose that $p_1a \equiv p_2a \;(\bmod\; p)$. Then $p_1a - p_2a  \equiv 0 \;(\bmod\; p) \Longrightarrow p \mid (p_1a - p_2a) \Longrightarrow p \mid a(p_1 - p_2)$. Since $(p, a) = 1$, we can conclude that $p \mid (p_1 - p_2)$. However, since $p_1$ and $p_2 \in P$, then $(p_1 - p_2)$ must be smaller then p. Therefore, $p \nmid (p_1 - p_2)$. This implies that every element in P is not congruent to each other modulo o. Hence they are all distinct elements as modulo p.\\

Now, let's consider the set $\{a,2a,3a,...,pa\}$. The same implication above follows the as well, as;
\begin{align*}
    &\Longrightarrow p_1a \equiv p_2a \;(\bmod\; p)& \text{$p_1$ and $p_2 \in P$}\\
    &\Longrightarrow p_1 \equiv p_2 \;(\bmod\; p)&\\
    &\Longrightarrow p_1 - p_2 \equiv 0 \;(\bmod\; p)&\\
    &\Longrightarrow p \mid (p_1 - p_2)a&\\
    &\Longrightarrow p \mid (p_1 - p_2)& \text{Since, $(p, a) = 1$}.
\end{align*}
However, since $p_1$ and $p_2 \in P$, then $(p_1 - p_2)$ must be smaller then p. Each element in the set $\{a,2a,3a,...,pa\}$ is distinct modulo p and thus is a complete residue system modulo p.
\end{proof}

\subsection*{4.14 Theorem} 
\quad \textit{Let p be a prime and let a be an integer not divisible by p. Then}
\begin{center}
    $a \cdot 2a \cdot 3a \cdot \cdot \cdot \cdot (p-1)a \equiv 1 \cdot 2 \cdot 3 \cdot \cdot \cdot \cdot (p-1) \;(\bmod\; p)$
\end{center}

\begin{proof}
Let p be a prime where $(p, a) = 1$ with $a \in \mathbf{Z}$. Let $a \cdot 2a \cdot 3a \cdot \cdot \cdot \cdot (p-1)a$ be multiples of a. Also lets assume that $xa, ya$ are multiples of a and that $xa \equiv ya \;(\bmod\; p)$, then;
\begin{align*}
    &xa \equiv ya \;(\bmod\; p)&\\
    &xa - ya \equiv 0 \;(\bmod\; p)&\\
    &p \mid (xa - ya)&\\
    &p \mid (x - y)a&\\
    &p \mid (x - y)& \text{Since, $(p, a) = 1$}.\\
    &x - y \equiv 0 \;(\bmod\; p)&\\
    &x \equiv y \;(\bmod\; p)&
\end{align*}
Therefore, $a \cdot 2a \cdot 3a \cdot \cdot \cdot \cdot (p-1)a$ are distinct multiples of a. They must be congruent to $1 \cdot 2 \cdot 3 \cdot \cdot \cdot \cdot (p-1)$ in some order $\Longrightarrow na \equiv a_n \;(\bmod\; p), n = 1,2,3,...,p-1$ where $a_n \in {1,2,3,...,p-1}$. Thus, the claim $a \cdot 2a \cdot 3a \cdot \cdot \cdot \cdot (p-1)a \equiv 1 \cdot 2 \cdot 3 \cdot \cdot \cdot \cdot (p-1) \;(\bmod\; p)$ is true.
\end{proof}

\subsection*{4.15 Theorem (Fermat's Little Theorem, Version I)} 
\quad \textit{If p is a prime and a is an integer relatively prime to p, then $a^{(p-1)} \equiv 1 \;(\bmod\; p)$.}

\begin{proof}
Given a prime p, let the set $\{1,2,3,...,p-1\}$ be all the number modulo p, by Theorem 4.14. We can claim that some integer a is in $\{1,2,3,...,p-1\}$ since $(p, a) = 1$.\\
If we multiply the numbers in the set $\{1,2,3,...,p-1\}$ by a, we obtain $\{a,2a,3a,...,(p-1)a\}$ and we know by Theorem 4.14 that;
\begin{align*}
    &&a \cdot 2a \cdot 3a \cdot \cdot \cdot \cdot (p-1)a &\equiv 1 \cdot 2 \cdot 3 \cdot \cdot \cdot \cdot (p-1) \;(\bmod\; p)&&\\
    &&a^{p-1} \cdot (p-1)! &\equiv (p-1)! \;(\bmod\; p)&&\\
    &&a^{p-1} &\equiv 1 \;(\bmod\; p)&& \text{Since, }(p-1)! \;(\bmod\; p) \equiv 1
\end{align*}
Since we know that all values in $\{1,2,3,...,p-1\}$ are all modulo p, thus $(p-1)! \;(\bmod\; p) \equiv 1$. Hence, $a^{(p-1)} \equiv 1 \;(\bmod\; p)$.holds.
\end{proof}

\subsection*{4.16 Theorem (Fermat's Little Theorem, Version II)} 
\quad \textit{If p is a prime and a is any integer; then $a^{p} \equiv a \;(\bmod\; p)$.}

\begin{proof}
Let p be a prime and $a \in \mathbf{Z}$. Assume that $a^{p} \equiv a \;(\bmod\; p)$, thus we need to prove that p divides $a^{p} - a$ which can be expressed as $p \mid a^{p} - a \Longrightarrow p \mid a(a^{p-1} - 1)$. Thus, we must either prove that $p \mid a$ or $p \mid a^{p-1} - 1$.\\
For $p \mid a^{p-1} - 1$, this can be proven by Theorem 4.15, if $(p, a) = 1$ or in other words $p \nmid a$. The theorem 4.15 fails if $p \mid a$. In that case where $p \mid a$, then $p \mid a(a^{p-1} - 1)$. Thus $a^{p} \equiv a \;(\bmod\; p)$ holds.
\end{proof}

\subsection*{4.17 Theorem} 
\quad \textit{The two versions of Fermat's Little Theorem stated above are equivalent to one another, that is, each one can be deduced from the other.}

\begin{proof}
By direct proof, given that by theorem 4.15, let p is a prime and a is an integer relatively prime to p, then $a^{(p-1)} \equiv 1 \;(\bmod\; p)$;
\begin{align*}
    &&a^{(p-1)} &\equiv 1 \;(\bmod\; p)&&\\
    &&a^{p}a^-1 &\equiv 1 \;(\bmod\; p)&&\\
    &&a^{p}a^-1 \cdot a &\equiv a \;(\bmod\; p)&& \text{Multiplying Both sides with the inverse of $a^{-1}$}\\
    &&a^{p} &\equiv a \;(\bmod\; p)&&
\end{align*}
Thus both theorems are the same.
\end{proof}

\subsection*{4.18 Theorem} 
\quad \textit{If p is a prime and a be an integer. If $(a,p) = 1$, then $ord_p(a)$ divides $p-1$, that is, $ord_p(a) \mid p-1$.}

\begin{proof}
Let p is a prime and a be an integer where $(a,p) = 1$.
\end{proof}

\subsection*{4.19 Exercise} 
\quad \textit{Compute each of the following without the aid of a calculator or computer.}

\begin{enumerate}
    \item $512^{372} \;(\bmod\; 13)$
          $512^{12 \cdot 13} \;(\bmod\; 13)$\\
          $512^{12 \cdot 13} \;(\bmod\; 13)$\\
          $512^12^13 \;(\bmod\; 13)$\\
          $1^13 \;(\bmod\; 13)$\\
          $1 \;(\bmod\; 13)$
    \item $3444^{3233} \;(\bmod\; 17)$\\
          $3444^{202(16) + 1} \;(\bmod\; 17)$\\
          $3444^{202(16)}3444^1 \;(\bmod\; 17)$\\
          $3444^1 \;(\bmod\; 17)$
    \item $123^{456} \;(\bmod\; 23)$
          $123^{20(22) + 16} \;(\bmod\; 23)$\\
          $123^{20(22)}123^16 \;(\bmod\; 23)$\\
          $123^16 \;(\bmod\; 23)$
\end{enumerate}

\subsection*{4.20 Exercise} 
\quad \textit{Find the remainder upon division of $314^{159}$ by 31.}

Using Fermat's Little Theorem, we can deduce that $314^{30} \equiv 1 \;(\bmod\; 31)$. Also we can deduce that $159 = 30(5) + 9$. Now,

\begin{align*}
    &\Longrightarrow 314^{159} \;(\bmod\; 31)&\\
    &\Longrightarrow 314^{30(5) + 9} \;(\bmod\; 31)&\\
    &\Longrightarrow 314^{30(5)}314^9 \;(\bmod\; 31)&\text{By Fermat's Little Theorem}\\
    &\Longrightarrow 314^9 \;(\bmod\; 31)&\\
\end{align*}

Now, we can calculate that $314 \equiv 4 \;(\bmod\; 31)$ by $314 = 31(10) + 4$. Therefore, we can manipulate $314 \equiv 4 \;(\bmod\; 31)$ to achieve $314^9 \;(\bmod\; 31)$;

\begin{align*}
    &&314 & \equiv 4 \;(\bmod\; 31)&&\\
    &&314^3 & \equiv 4^3 \;(\bmod\; 31)&&\\
    &&314^3 & \equiv 2 \;(\bmod\; 31)&&\\
    &&314^{3\codt3} & \equiv 2^3 \;(\bmod\; 31)&&\\
    &&314^9 & \equiv 8 \;(\bmod\; 31)&&
\end{align*}

\subsection*{4.21 Theorem} 
\quad \textit{Let n and m be natural numbers that are relatively prime, and let a be an integer. If $x \equiv a \;(\bmod\; n)$ and $x \equiv a \;(\bmod\; m)$, then $x \equiv a \;(\bmod\; nm)$.}

\begin{proof}
By Theroem 2.25, let a, n, and m be integers. If $n \mid (x-a)$, $m \mid (x-a)$, and $(n,m) = 1$, then $nm, \mid (x-a)$.

Also can be proven by the Chinese Remainder Theorem.
\end{proof}

\subsection*{4.22 Exercise} 
\quad \textit{Find the remainder when $4^{72}$ is divided by $91 (= 7 \cdot 13)$.}

We are attempting to solve $4^{72} \;(\bmod\; 91)$. We can deduce that $72 = (7-1)(13-1) = 6 * 12$. By Theorem, 4.21, we can express this as $x \equiv 4^{72} \;(\bmod\; 7)$ and $x \equiv 4^{72} \;(\bmod\; 13)$. Thus,
\begin{align*}
    && x &\equiv 4^{72} \;(\bmod\; 7)&&\\
    && x &\equiv 4^{6\cdot 12} \;(\bmod\; 7)&&\\
    && x &\equiv 1^12 \;(\bmod\; 7)&&\\
    && x &\equiv 1 \;(\bmod\; 7)&&
\end{align*}
Similarly,
\begin{align*}
    && x &\equiv 4^{72} \;(\bmod\; 13)&&\\
    && x &\equiv 4^{6\cdot 12} \;(\bmod\; 13)&&\\
    && x &\equiv 1^6 \;(\bmod\; 13)&&\\
    && x &\equiv 1 \;(\bmod\; 13)&&
\end{align*}

\subsection*{4.23 Exercise} 
\quad \textit{Find the natural number $k < 117$ such that $2^{117} \equiv k \;(\bmod\; 117)$. (117 is not a prime).}

\begin{align*}
    && 2^{117} &\equiv k \;(\bmod\; 117) &&\\
    && 2^{13 \cdot 3^2} &\equiv k \;(\bmod\; 13 \cdot 3^2) &&
\end{align*}

By the Chinese Remainder Theorem, $2^{117} \equiv 2^3 \equiv 8 \;(\bmod\; 9)$ and $2^{117} \equiv 2^9 \equiv 5 \;(\bmod\; 13)$.\\
Now let $M = m_1m_2m_3 = 3 \cdot 3 \cdot 13 = 117$.\\
\begin{itemize}
    \item $M_1 = \frac{M}{m_1} = \frac{117}{3} = 39y_1 \equiv 1 \;(\bmod\; 3)$
    \item $M_2 = \frac{M}{m_2} = \frac{117}{3} = 39y_2 \equiv 1 \;(\bmod\; 3)$
    \item $M_3 = \frac{M}{m_3} = \frac{117}{13} = 9y_3 \equiv 1 \;(\bmod\; 13)$
\end{itemize}

Now, we need to find the Multiplication inverse of each;
\begin{itemize}
    \item $39y_1 \equiv 1 \;(\bmod\; 3)$ thus $39y_1 \equiv 1 \;(\bmod\; 3), y_1 = 1$
    \item $39y_2 \equiv 1 \;(\bmod\; 3)$ thus $39y_2 \equiv 1 \;(\bmod\; 3), y_2 = 1$
    \item $9y_3 \equiv 1 \;(\bmod\; 13)$ thus $9y_3 \equiv 1 \;(\bmod\; 13), y_3 = 3$
\end{itemize}

Therefore;
\begin{center}
    $k = 8(39) + 8(39) + 5(9\cdot 3) = 795 \equiv 93 \;(\bmod\; 117)$
\end{center}

\subsection*{4.24 Theorem} 
\quad \textit{Let a and b be numbers and let n be a natural number. Then}
\begin{center}
    $(a+b)^n = \sum_{i = 0}^{n} \binom{n}{i} a^{n-i}b^i$
\end{center}

\subsection*{4.25 Lemma} 
\quad \textit{If p is prime and i is a natural number less than p, then p divides $\binom{p}{i}$.}

\begin{proof}
Given p prime and i, a natural number and less than p, we can expand $\binom{p}{i}$;
\begin{align*}
    &&\binom{p}{i} &= \frac{p!}{i!(p-i)!}&&\\
    &&\binom{p}{i} &= \frac{p(p-1)(p-2)\cdot\cdot\cdot(p-(i-1))(p-i)!}{i!(p-i)!}&&\\
    &&\binom{p}{i} &= \frac{p(p-1)(p-2)\cdot\cdot\cdot(p-(i-1))}{i!}&&\\
    &&\binom{p}{i}\cdot i! &= p(p-1)(p-2)\cdot\cdot\cdot(p-(i-1))&&
\end{align*}
Now, p divides p thus we can express it as the following;
\begin{center}
    $p \mid p \Longrightarrow p \mid p(p-1)(p-2)\cdot\cdot\cdot(p-(i-1))$
    $\Longrightarrow p \mid \binom{p}{i}\cdot i!$
    $\Longrightarrow p \mid \binom{p}{i}\cdot i(i-1)(i-2)\cdot\cdot\cdot3\cdot2\cdot1$
\end{center}
Since i is less than p, p can not divide $i!$. Thus we end up with $p \mid \binom{p}{i}$. Therefore, p must divide $\binom{p}{i}$.
\end{proof}

\subsection*{4.26 Theorem (Fermat's Little Theorem, Version II)} 
\quad \textit{If p is a prime and a is any integer; then $a^{p} \equiv a \;(\bmod\; p)$.}

\begin{proof}
\end{proof}

\subsection*{4.27 Question} 
\quad \textit{The numbers 1,5,7 and 11 are all natural numbers less than or equal to 12 that are relatively prime to 12, so $\phi(12) = 4.$}

\begin{enumerate}
    \item What is $\phi(7)$? \textbf{$\phi(7) = 4$}
    \item What is $\phi(15)$? \textbf{$\phi(15) = 8$}
    \item What is $\phi(21)$? \textbf{$\phi(21) = 12$}
    \item What is $\phi(35)$? \textbf{$\phi(35) = 24$}
\end{enumerate}

\subsection*{4.28 Theorem} 
\quad \textit{Let a, b, and n be integers such that $(a,n) = 1$ and $(b,n) = 1$. Then $(ab, n) = 1$.}

\begin{proof}
By Theorem 1.43.
\end{proof}

\subsection*{4.29 Theorem} 
\quad \textit{Let a, b, and n be integers with $n > 0$. If $a \equiv b \;(\bmod\; n)$ and $(a,n)=1$, then $(b,n)=1$.}

\begin{proof}
Given a, b, and n be integers with $n > 0$, $a \equiv b \;(\bmod\; n)$ and $(a,n)=1$. We can express $a \equiv b \;(\bmod\; n) \Longrightarrow n \mid (a - b) \Longrightarrow nd = a -b \Longrightarrow a = nd + b, \exists d \in \mathbf{Z}$. Since $(a, n) = 1$, $b \neq 0$. Also, $(a, n) = 1 = ax + ny, \exists x, y \in \mathbf{Z}$.\\

Now, we can substitute in $a = nd + b$ for a in $1 = ax + ny$;
\begin{align*}
    && 1 &= ax + ny &&\\
    && 1 &= (nd + b)x + ny &&\\
    && 1 &= ndx + bx + ny &&\\
    && 1 &= n(dx + y) + bx&&\\
    && 1 &= nz + bx&& \text{Some integer $z = dx+y$}\\
    && 1 &= (b, n)&&
\end{align*}
\end{proof}

\subsection*{4.30 Theorem} 
\quad \textit{Let a, b, c and n be integers with $n > 0$. If $ac \equiv bc \;(\bmod\; n)$ and (c,n) = 1, then $a \equiv b \;(\bmod\; n)$.}

\begin{proof}
By Theorem 1.45.
\end{proof}

\subsection*{4.31 Theorem} 
\quad \textit{Let n be a natural number and let $x_1, x_2, ...,x_{\phi(n)}$ be the distinct natural numbers less than or equal to n that are relatively prime to n. Let a be a non-zero integer relatively prime to n and let i and j be different natural numbers less than or equal to $\phi(n)$. Then $ax_i \not\equiv ax_j \;(\bmod\; n)$.}

\begin{proof}
Let n be a natural number and let $x_1, x_2, ...,x_{\phi(n)}$ be the distinct natural numbers less than or equal to n that are relatively prime to n. Also let a be a non-zero integer relatively prime to n and let i and j be different natural numbers less than or equal to $\phi(n)$.\\
Now we must show that $ax_i \not\equiv ax_j \;(\bmod\; n)$. Suppose, by contradiction, $ax_i \equiv ax_j \;(\bmod\; n)$, we can express this as $n \mid a(x_i - x_j)$. Since $(a, n) = 1$, then $n \mid (x_i - x_j)$. By definition, that also signifies that $x_i \equiv x_j \;(\bmod\; n)$ which is impossible since we have stated that i and j are different natural numbers. Thus, $x_i \not\equiv x_j \;(\bmod\; n)$ which also implies that $ax_i \not\equiv ax_j \;(\bmod\; n)$.
\end{proof}

\subsection*{4.32 Theorem (Euler's Theorem)} 
\quad \textit{If a and n are integers with $n > 0$ and $(a, n) = 1$, then}
\begin{center}
    $a^{\phi(n)} \equiv 1 \;(\bmod\; n)$
\end{center}

\begin{proof}
Let a and n be integers with $n > 0$ and $(a,n) = 1$. Also suppose a set of integers $A = \{x_1, x_2,...,x_{\phi_n}\}$ less than n such that $(n, x_i) = 1, i = 1,2,3,...,n$.\\
Then, we'll need to prove that;
\begin{center}
    $\{ax_1 \;(\bmod\; n), ax_2 \;(\bmod\; n), ax_3 \;(\bmod\; n),..., ax_{\phi_n} \;(\bmod\; n)\}$\\
    $ = \{x_1, x_2,...,x_{\phi_n}\}$
\end{center}
We can observe that each $ax_i \;(\bmod\; n)$ is less than n and we know that $(n, x_i) = 1$ and that $(n, a) = 1$, thus, by Theorem 4.28, $(n, ax_i) = 1$. Which then implies that the following is true;
\begin{center}
    $\{ax_1 \;(\bmod\; n), ax_2 \;(\bmod\; n), ax_3 \;(\bmod\; n),..., ax_{\phi_n} \;(\bmod\; n)\}$\\
    $ = \{x_1, x_2,...,x_{\phi_n}\}$
\end{center}
This implies then that;
\begin{align*}
    &&ax_1 \cdot ax_2 \cdot ax_3 \cdot \cdot\cdot ax_{\phi_n} &\equiv x_1 \cdot x_2 \cdot x_3 \cdot \cdot\cdot x_{\phi_n} \;(\bmod\; n)&&\\
    &&a^{\phi(n)}(x_1 \cdot x_2 \cdot x_3 \cdot \cdot\cdot x_{\phi_n}) &\equiv x_1 \cdot x_2 \cdot x_3 \cdot \cdot\cdot x_{\phi_n} \;(\bmod\; n)&&\\
    &&a^{\phi(n)} &\equiv 1 \;(\bmod\; n)&&
\end{align*}
Thus, the Euler Theorem holds.
\end{proof}

\subsection*{4.33 Corollary (Fermat's Little Theorem)} 
\quad \textit{If p is a prime and a is an integer relatively prime to p, then $a^{p-1} \equiv 1 \;(\bmod\; p)$.}

\begin{proof}
We can use another form of Fermat's Little Theorem which is $a^p \equiv a \;(\bmod\; p)$ with a p prime and an integer a such that $(p,a) = 1$. Now, we can deduce that the $\phi(p) = p-1$ since primes are not divisible by any other number. Thus, we can express Fermat's Little Theorem as $a^{\phi(p)} \equiv a \;(\bmod\; p)$. Since $(p,a) = 1$ an by theorem 4.2, we can then conclude that $a^{\phi(p)} \equiv a \;(\bmod\; p)$ holds.
\end{proof}

\subsection*{4.34 Exercise} 
\quad \textit{Compute each of the following without the aid of a calculator or computer.}

\begin{enumerate}
    \item $12^{49} \;(\bmod\; 15)$ with $\phi(15) = 8$\\
          $12^{7\cdot 7} \;(\bmod\; 15)$\\
          $12^{(8-1)\cdot 7} \;(\bmod\; 15)$\\
          $(12^812^{-1})^7 \;(\bmod\; 15)$\\
          $(12^{-1})^7 \;(\bmod\; 15)$\\
          $12^{-8+1} \;(\bmod\; 15)$\\
          $12^{-8}12^1 \;(\bmod\; 15)$\\
          $12^1 \;(\bmod\; 15)$\\
          $12 \;(\bmod\; 15)$\\
    \item $139^{112} \;(\bmod\; 27)$ with $\phi(27) = 18$\\
          $139^{18(7) + 13} \;(\bmod\; 27)$\\
          $139^{18(7)}139^{13} \;(\bmod\; 27)$\\
          $139^13 \;(\bmod\; 27)$\\
          $139^{18-5} \;(\bmod\; 27)$\\
          $139^5 \;(\bmod\; 27)$
\end{enumerate}

\subsection*{4.35 Exercise} 
\quad \textit{Find the last digit in the base 10 representation of the integer $13^{474}$.}

\begin{itemize}
    \item $13 \equiv 3 \;(\bmod\; 10)$
    \item $13^0 \equiv 3^0 \equiv 1 \;(\bmod\; 10)$
    \item $13^2 \equiv 3^2 \equiv 9 \;(\bmod\; 10)$
    \item $13^3 \equiv 3^3 \equiv 7 \;(\bmod\; 10)$
    \item $13^4 \equiv 3^4 \equiv 1 \;(\bmod\; 10)$
    \item Thus we can find the exponent remainder at modulo 4.
    \item $474 \equiv 2 \;(\bmod\; 4)$
    \item Thus, $13^2 \equiv 169 \equiv 9 \;(\bmod\; 10)$
\end{itemize}

\subsection*{4.36 Theorem} 
\quad \textit{Let p be a prime and let a be an integer such that $1 \leq a < p$. Then there exists a unique natural number b less than p such that $ab \equiv 1 \;(\bmod\; p)$.}

\begin{proof}
Let p be a prime and let a be an integer such that $1 \leq a < p$. Assume that there exists a unique natural number b and less than p. We can deduce that $1 \leq a < p$ implies that $(a, p) = 1$ since p is prime and a is less than p. Moreover, we can also express $ab \equiv 1 \;(\bmod\; p)$ as $(ab, p) = 1$. Now by Theorem 4.29, for $(ab, p) = 1$ to hold, $(b, p) = 1$ must also hold. In this case we know than b is unique and is less than prime p, thus b can not divide p or p can not divide b. Therefore, $(b, p) = 1$ must be true and thus so is $(ab, p) = 1$.
\end{proof}

\subsection*{4.37 Exercise} 
\quad \textit{Let p be a prime. Show that the natural numbers 1 and $p-1$ are their own inverses modulo p.}

\begin{proof}
Let p be a prime number. Let also the set $\{1,2,3,...,(p-1)\}$. We can show that 1 is it's own inverse since 1 holds the identity property of the aforementioned set, $1^{-1} \equiv 1 \;(\bmod\; p)$. Now for $p-1$, we'll first consider $(p-1)(p-1)$,
\begin{align*}
    && (p-1)^2 &= p^2 -2p + 1 &&\\
    && (p-1)^2 &= p(p - 2) + 1 &&\\
    && (p-1)^2 \;(\bmod\; p) &= (p(p - 2) + 1) \;(\bmod\; p)&& \text{Apply $\;(\bmod\; p)$ in both sides}\\
    && (p-1)^2 &\equiv 1 \;(\bmod\; p)&&\\
    && (p-1)(p-1) &\equiv 1 \;(\bmod\; p)&&\\
    && (p-1)(p-1)\cdot(p-1)^{-1} &\equiv (p-1)^{-1} \;(\bmod\; p)&&\\
    && (p-1) &\equiv (p-1)^{-1} \;(\bmod\; p)&&
\end{align*}
Thus, the natural numbers 1 and $p-1$ are their own inverses modulo p.
\end{proof}

\subsection*{4.38 Theorem} 
\quad \textit{Let p be a prime and let a and b be integers such that $1 < a, b < p-1$ and $ab \equiv 1 \;(\bmod\; p)$. Then $a \neq b$.}

\begin{proof}
Let p be a prime and let a and b be integers such that $1 < a, b < p-1$ and $ab \equiv 1 \;(\bmod\; p)$. Now lets assume, by contradiction, that $a = b$. Then, $ab \equiv a^2 \equiv 1 \;(\bmod\; p)$. Also, by definition, $a^2 \equiv 1 \;(\bmod\; p) \Longrightarrow p \mid a^2 -1$ where $a^2-1 = (a+1)(a-1)$. Since p is a prime, can divide $a-1$ or $a+1$.\\
However, since we know that $1 < a < p-1$, which can also be expressed as $1 < a + 1< p$, this implies that p can not divide $a_1$. Similarly, since $a < p-1$, p can not divide $a+1$ either. This signifies that $p \nmid a^2 = ab, a = b$. Thus $a \neq b$. 
\end{proof}

\subsection*{4.39 Exercise} 
\quad \textit{Find all pairs of number a and b in $\{2,3,...,11\}$ such that $ab \equiv 1 \;(\bmod\; 13)$.}

$ab \equiv 1 \;(\bmod\; 13)$ implies that $a^{-1} \equiv b \;(\bmod\; 13)$. Now with the set of all natural numbers less than 13 $\{2,3,4,5,6,7,8,9,10,11,12\}$. Let's start to identify pairs a an b where $ab \equiv 1 \;(\bmod\; 13)$ implies that $a^{-1} \equiv b \;(\bmod\; 13)$.
\begin{itemize}
    \item 2 and 7
    \item 3 and 9
    \item 5 and 8
    \item 4 and 10
    \item 6 and 11
\end{itemize}

\subsection*{4.40 Theorem} 
\quad \textit{If p is a prime larger than 2, then $2 \cdot 3 \cdot 4 \cdot\cdot\cdot\cdot(p-2) \equiv 1 \;(\bmod\; p)$.}

\begin{proof}
Let there be a prime p larger than 2. Then each integer from 1,...,(p-1) has an inverse modulo p. Now we must show that the inverse is unique.\\
Let a, b and c be integers where a has no inverse for some distinct b and c modulo p;
\begin{align*}
    && ab &\equiv ac \;(\bmod\; p)&&\\
    && ab - ac &\equiv 0 \;(\bmod\; p)&&\\
    && a(b - c) &\equiv 0 \;(\bmod\; p)&&
\end{align*}
Therefore, $p \mid a(b-c)$. However, p does not divide a or b-c, which is a contraction. Thus, 1,...,(p-1) has unique inverse modulo p.\\
If a is it's own inverse then;
\begin{align*}
    && a^2 &\equiv 1 \;(\bmod\; p)&&\\
    && a^2-1 &\equiv 0 \;(\bmod\; p)&&\\
    && (a+1)(a-1) &\equiv 0 \;(\bmod\; p)&&
\end{align*}
Thus, $a \equiv 1 \;(\bmod\; p)$ or $a \equiv -1 \;(\bmod\; p) \Longrightarrow a \equiv p-1 \;(\bmod\; p)$. Hence, we can conclude that the inverse of 1 and p-1 is a itself.\\
Thus, for each integer $a \in \{2,...,(p-2)\}$ there exist a $b \in \{2,...,(p-2)\}$ such that $ab\equiv 1 \;(\bmod\; p)$ which will get $2 \cdot 3 \cdot 4 \cdot\cdot\cdot\cdot(p-2) \equiv 1 \;(\bmod\; p)$.
\end{proof}

\subsection*{4.41 Theorem (Wilson's Theorem)} 
\quad \textit{If p is a prime, then $(p-1)! \equiv -1 \;(\bmod\; p)$.}

\begin{proof}
By Theorem, 4.40, suppose p is a prime larger than 2, then $2 \cdot 3 \cdot 4 \cdot\cdot\cdot\cdot(p-2) \equiv 1 \;(\bmod\; p)$. Now, let's multiply (p-1) in both sides;
\begin{align*}
    &&2 \cdot 3 \cdot 4 \cdot\cdot\cdot\cdot(p-2) &\equiv 1 \;(\bmod\; p) &&\\
    &&2 \cdot 3 \cdot 4 \cdot\cdot\cdot\cdot(p-2)\cdot (p-1) &\equiv (p-1) \;(\bmod\; p) &&\\
    &&2 \cdot 3 \cdot 4 \cdot\cdot\cdot\cdot(p-2)\cdot (p-1) &\equiv -1 \;(\bmod\; p) &&\\
    &&(p-1)! &\equiv -1 \;(\bmod\; p) &&
\end{align*}
Thus, if p is a prime, then $(p-1)! \equiv -1 \;(\bmod\; p)$.
\end{proof}

\subsection*{4.42 Theorem (Converse of Wilson's Theorem)} 
\quad \textit{If n is a natural number such that}
\begin{center}
    $(n-1)! \equiv -1 \;(\bmod\; n)$,
\end{center}
\textit{then n is prime.}

\begin{proof}
Given that n is a natural number such that $(n-1)! \equiv -1 \;(\bmod\; n)$. By contradiction let's assume that n is composite. Therefore, it's possible divisions are $1,2,...,n-1$ which also implies that the $gcd((n-1)!, n) > 1$. If that's the case then we can not have $(n-1)! \equiv -1 \;(\bmod\; n)$. Therefore, n must be prime.
\end{proof}

\subsection*{4.43 Blank Paper Exercise} 
\begin{itemize}
    \item Integer to the high power modulo n 
    \item Fermat's Little Theorem
    \item Binomial Theorem
    \item Euler's Theorem
    \item Wilson's Theorem
    \item Proof 4.24 and 4.26 were extremely difficult and I wasn't able to get.
\end{itemize}

\section*{Supplementary Exercises}
\subsection*{4.1.1 Exercise} 

\begin{itemize}
    \item a = 2
    \begin{itemize}
        \item $2^2 \;(\bmod\; 3) \equiv 1$
        \item $2^3 \;(\bmod\; 3) \equiv 2$
        \item $2^5 \;(\bmod\; 6) \equiv 1$
        \item $2^7 \;(\bmod\; 3) \equiv 2$
        \item $2^8 \;(\bmod\; 9) \equiv 1$
    \end{itemize}
    \item a = 3
    \begin{itemize}
        \item $3^2 \;(\bmod\; 5) \equiv 4$
        \item $3^3 \;(\bmod\; 5) \equiv 2$
        \item $3^5 \;(\bmod\; 6) \equiv 1$
        \item $3^7 \;(\bmod\; 5) \equiv 2$
        \item $3^8 \;(\bmod\; 9) \equiv 1$
    \end{itemize}
\end{itemize}

\textbf{Conjecture.} \textit{When a is a prime 2 or 3 and k=n-1 where n is prime, then $a^{k} \equiv 1 \;(\bmod\; n)$.}

\subsection*{4.1.2 Theorem} 
\quad \textit{If $(a,n) = 1$, the period d of the sequence $a^k \;(\bmod\; n)$ is given by $d = ord_n(a)$.}

\begin{proof}
By Theorem 4.6.
\end{proof}

\subsection*{4.1.3.a Theorem} 
\quad \textit{If $n = a^k$, the sequence of $a^{k} \equiv 1 \;(\bmod\; n)$ values becomes all zeros after an initial k − 1 terms, and these are the only cases in which the period of repetition is one.}

\begin{proof}
By definiton, let a and n be natural numbers with $(a,n) = 1$. The smallest natural number k such that $a^k \equiv 1 \;(\bmod\; n)$ is $ord_n(a)$.
\end{proof}

\subsection*{4.1.3.b Theorem} 
\quad \textit{If $n = a^k − 1$ then $d = k$ and if $n = a^k + 1$, then $d = 2k$.}

\begin{proof}
Since $n = a^k − 1$ can imply that $a^{k} \equiv 1 \;(\bmod\; n)$ then if $n = a^k + 1$ implies that $a^{k} + 2 \equiv 1 \;(\bmod\; n)$, thus $d = 2k$.
\end{proof}

\subsection*{4.1.4 Theorem} 
\quad \textit{If q is prime with (a, q) = 1 and $n = a^rq$ for some natural number r, define $ord_n(a) = min{k | a^k = 1 mod n}$. Then $ord_n(a) = ord_q(a)$ and q = 1 mod (k).}

\begin{proof}
By definiton, let a and n be natural numbers with $(a,n) = 1$. The smallest natural number k such that $a^k \equiv 1 \;(\bmod\; n)$ is $ord_n(a)$. Thus if $ord_n(a) = min{k | a^k = 1 mod n}$ then $ord_n(a) = ord_q(a)$ if and only if q = 1 mod (k).
\end{proof}

\subsection*{4.1.5 Question}
That the order will not still signify the smallest value. As there will be instances where $a^k \equiv 0 \;(\bmod\; n)$.

\subsection*{4.1.6 Question}
That if (a, n) $\neq 1$ then the order will still be less than n and signify the smallest possible value b such that $a^k \equiv b \;(\bmod\; n)$.

\subsection*{4.1.7 Question}
We are able to define the order function for instances where a and n are not relatively prime.

\subsection*{4.2.1 Lemma}
\begin{proof}
In the case where n is prime, we can prove this theorem by Lemma 4.25. However, when n is composite we can notice that n always divides \binom{n-1}{i}. 
\end{proof}

\subsection*{4.2.2 Lemma}
\begin{proof}
\begin{tabular}{>{$n=}l<{$\hspace{12pt}}*{13}{c}}
0 &&&&&&&1&&&&&&\\
1 &&&&&&1&&1&&&&&\\
2 &&&&&1&&2&&1&&&&\\
3 &&&&1&&3&&3&&1&&&\\
4 &&&1&&4&&6&&4&&1&&\\
5 &&1&&5&&10&&10&&5&&1&\\
6 &1&&6&&15&&20&&15&&6&&1
\end{tabular}
\end{proof}

\subsection*{4.2.3 Theorem}
\quad \textit{For all primes p and all $k \geq 0, (a+b)^{p^k} = {a^{p^k}} + b^{p^k} \;(\bmod\; p)$.}

\begin{proof}
$(a+b)^{p^k}$ will always result to ${a^{p^k}} + ... + b^{p^k}$. Since everything in between will be a product of a and b of some sort, when we perform modulo p, it will result to 0, since p is then considered for all primes.
\end{proof}

\subsection*{4.2.4 Exercise}
\begin{itemize}
    \item $(a+b)^{17} \equiv a^{17} + b^{17} \;(\bmod\; 2)$
    \item $(a+b)^{585} \equiv a^{585} + b^{585} \;(\bmod\; 2)$
    \item $(a+b)^{279} \equiv a^{279} + b^{279} \;(\bmod\; 2)$
    \item $(a+b)^{3155} \equiv a^{3155} + b^{3155} \;(\bmod\; 2)$
\end{itemize}

\subsection*{4.2.7 Theorem}
\quad \textit{For $n = p^k, (a_1 + · · · + a_n)^n = a_1^n + · · · + a_m^n$ mod (p). This result is especially useful for p = 2.}

\begin{proof}
By Theorem 4.2.3.
\end{proof}

\end{document}