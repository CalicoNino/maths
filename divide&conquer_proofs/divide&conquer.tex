\documentclass{article}
\usepackage[utf8]{inputenc}
\usepackage[english]{babel}

\usepackage{amsthm}
\usepackage{amsmath}
\usepackage{amssymb}

\title{Divide \& Conquer Proofs}
\author{Karim El Shenawy}
\date{January 2021}

\usepackage{natbib}
\usepackage{graphicx}

\begin{document}

\maketitle

\section*{Introduction}
This course notebook is the collection of theorem proofs, exercises and answers from Unit 1 of the Number Theory Through Inquiry (Mathematical Association of America Textbooks).

\section*{Divisibility \& Congruence}

\subsection*{1.1 Theorem} 
\quad \textit{Let a, b, and c be integers. If $a \vert b$ and $a \vert c$, then $a \vert (b+c)$.}

\begin{proof}
Our hypothesises that $a \vert b$ and $a \vert c$ both mean respectively, by definition, that $b = k_1a$ and $c = k_2a$ for some integer $k_1$ and $k_2$. Also by definition, $a \vert (b+c)$ means that $b + c = k_3a$ for some integer $k_3$. Using that we can say that $b + c = k_1a + k_2a = a(k_1+k_2) = k_3a$. Thus by definition, $a \vert a(k_1+k_2)$, knowing that $(b+c) = a(k_1+k_2)$, we can satisfy the definition of $a \vert (b+c)$.
\end{proof}

\subsection*{1.2 Theorem} 
\quad \textit{Let a, b, and c be integers. If $a \vert b$ and $a \vert c$, then $a \vert (b-c)$.}

\begin{proof}
Our hypothesises that $a \vert b$ and $a \vert c$ both mean respectively, by definition, that $b = k_1a$ and $c = k_2a$ for some integer $k_1$ and $k_2$. Also by definition, $a \vert (b-c)$ means that $b - c = k_3a$ for some integer $k_3$. Using that we can say that $b - c = k_1a - k_2a = a(k_1 - k_2) = k_3a$. Thus by definition, $a \vert a(k_1 - k_2)$, knowing that $(b - c) = a(k_1 - k_2)$, we can satisfy the definition of $a \vert (b - c)$.
\end{proof}

\subsection*{1.3 Theorem} 
\quad \textit{Let a, b, and c be integers. If $a \vert b$ and $a \vert c$, then $a \vert bc$.}

\begin{proof}
Our hypothesises that $a \vert b$ and $a \vert c$ both mean respectively, by definition, that $b = k_1a$ and $c = k_2a$ for some integer $k_1$ and $k_2$. Also by definition, $a \vert bc$ means that $bc = (k_1a)(k_2a) = k_1k_2a^2$. Using that we can say that $bc = k_1k_2a^2 = (k_1k_2a)a$. Thus by definition, $a \vert (k_1k_2a)$, knowing that $bc= (k_1k_2a)a$, we can satisfy the definition of $a \vert bc$.
\end{proof}

\subsection*{1.4 Question} 
\quad \textit{Can you weaken the hypothesis of the previous theorem and still prove the conclusion? Can you keep the same hypothesis, but replace the conclusion by the stronger conclusion that $a^2 \vert bc$ and still prove the theorem?}

We can weaken the hypothesis by saying that by definition, if $a \vert c$ is true then $a \vert kc$ for some integer k. We can then maintain the same hypothesis and also state that because $bc = (k_1a)(k_2a) = k_1k_2a^2$, $a^2 \vert bc$

\subsection*{1.5 Question} 
\quad \textit{Can you formulate your own conjecture along the lines of the above theorems and then prove it to make your theorem?}

We can formulate the following conjecture: \textit{Let a, b and c be integers. If $a \vert b$ and $a \vert c$ then $a \vert t$ where $t$ is the sum, difference or the multiplication total of $b$ and $c$.}
\begin{proof}
Using theorems 1.1, 1.2 and 1.3, we can proof this theorem.
\end{proof}

\subsection*{1.6 Theorem} 
\quad \textit{Let a, b, and c be integers. If $a \vert b$, then $a \vert bc$.}

\begin{proof}
By definition, we can deduce that $bc = ka$ where $k \in \mathbb{Z}$ since $a \vert b$ then $b = na$ where $n \in \mathbb{Z}$. Substituting b, 
\begin{flalign*}
    && bc &= ka &&\\
    && (na)c &= ka &&\\
    && a(nc) &= a(k) &&\\
    && (nc) &= (k)
\end{flalign*}
Which ca n be expressed as $c \vert k$. Therefore, we can conclude that $bc = ka$ is true.

\end{proof}

\subsection*{1.7 Exercise} 
\begin{enumerate}
    \item \textit{Is $45 \equiv 9 \;(\bmod\; 4)$?}
    \newline Yes, since $4 \vert (45-9) = 4 \vert 36$ and 4 does divide 36.
    \item \textit{Is $37 \equiv 2 \;(\bmod\; 5)$?}
    \newline Yes, since $5 \vert (37-2) = 5 \vert 35$ and 5 does divide 35.
    \item \textit{Is $37 \equiv 3 \;(\bmod\; 5)$?}
    \newline No, since $5 \vert (37-3) = 5 \vert 34$ and 5 does not divide 34.
    \item \textit{Is $37 \equiv -3 \;(\bmod\; 5)$?}
    \newline Yes, since $5 \vert (37-(-3)) = 5 \vert 40$ and 5 does divide 40.
\end{enumerate}

\subsection*{1.8 Exercise} 
\begin{enumerate}
    \item $m \equiv 0 \;(\bmod\; 3)$.
    \newline $m$ can be any integer such that $3 \vert m$ thus m can be any integer from \{$-3(N), -3(N-1)...-3(1), 3(1),3(2),3(3),12,15...3(N-1), 3(N)$\} where is $N$ is the length of the set.
    \item $m \equiv 1 \;(\bmod\; 3)$.
    \newline $m$ can be any integer such that $3 \vert (m+1)$ thus m can be any integer from \{$-3(N)-1, -3(N-1)-1...-3(1)-1, 3(1)-1,3(2)-1,3(3)-1,11,14...3(N-1)-1, 3(N)-1$\} where is $N$ is the length of the set.
    \item $m \equiv 2 \;(\bmod\; 3)$.
    \newline $m$ can be any integer such that $3 \vert (m+2)$ thus m can be any integer from \{$-3(N)-2, -3(N-1)-2...-3(1)-2, 3(1)-2,3(2)-2,3(3)-2,10,13...3(N-1)-2, 3(N)-2$\} where is $N$ is the length of the set.
    \item $m \equiv 3 \;(\bmod\; 3)$.
    \newline $m$ can be any integer such that $3 \vert m$ thus m can be any integer from \{$-3(N), -3(N-1)...-3(1), 3(1),3(2),3(3),12,15...3(N-1), 3(N)$\} where is $N$ is the length of the set.
    \item $m \equiv 4 \;(\bmod\; 3)$.
    \newline $m$ can be any integer such that $3 \vert m+4$ thus m can be any integer from \{$-3(N)-1, -3(N-1)-1...-3(1)-1, 3(1)-1,3(2)-1,3(3)-1,11,14...3(N-1)-1, 3(N)-1$\} where is $N$ is the length of the set.
\end{enumerate}

\subsection*{1.9 Theorem} 
\quad \textit{Let a, and n be integers with $n > 0$. Then $a \equiv a \;(\bmod\; n)$.}

\begin{proof}
By Definition, the statement above can be written as $n \vert a-a$ which would also mean that $n \vert 0$. And $n$ does divide $0$ with the following logic $0 = n * 0$.
\end{proof}

\subsection*{1.10 Theorem} 
\quad \textit{Let a, b, and n be integers with $n > 0$. If $a \equiv b \;(\bmod\; n)$, then $b \equiv a \;(\bmod\; n)$}

\begin{proof}
By definition, we can represent the above statements as $n \vert a-b$ and $n \vert b-a$. Using the Theorem 1.2 proved above we can deduce that this is true.
\end{proof}

\subsection*{1.11 Theorem} 
\quad \textit{Let a, b, c and n be integers with $n > 0$. If $a \equiv b \;(\bmod\; n)$ and $b \equiv c \;(\bmod\; n)$ then $a \equiv c \;(\bmod\; n)$}

\begin{proof}
By definition, $a \equiv b \;(\bmod\; n)$ can be represented as:
\begin{flalign*}
&& (a-b) &= nk && \text{for a given integer k} \\
\Rightarrow && 1) b &= a - nk
\end{flalign*}

Similarly, 
\begin{flalign*}
&& (c-b) &= ns && \text{for a given integer s} \\
\Rightarrow && 2) b &= c - ns
\end{flalign*}
Now equating 1) and 2),
\begin{flalign*}
&& c &= c && \\
\Rightarrow && a-nk &= c - ns && \\
\Rightarrow && a-c &= nk - ns && \\
\Rightarrow && a-c &= n(k - s)
\end{flalign*}
Thus, satisfying the claim that $a \equiv c \;(\bmod\; n)$.
\end{proof}

\subsection*{1.12 Theorem} 
\quad \textit{Let a, b, c, d and n be integers with $n > 0$. If $a \equiv b \;(\bmod\; n)$ and $c \equiv d \;(\bmod\; n)$ then $a + c \equiv b+d \;(\bmod\; n)$}

\begin{proof}
By definition, $a \equiv b \;(\bmod\; n)$ can be represented as:
\begin{flalign*}
&& (a-b) &= nk && \text{for a given integer k} \\
\Rightarrow && 1) b &= a - nk
\end{flalign*}

Similarly, 
\begin{flalign*}
&& (c-d) &= ns && \text{for a given integer s} \\
\Rightarrow && 2) d &= c - ns
\end{flalign*}
Now adding 1) and 2),
\begin{flalign*}
&& b+d &= (a - nk) + (c - ns) && \\
\Rightarrow && b+d &= a + c - n(k + s) && \\
\Rightarrow && n(k + s)&= (a + c) - (b+d)
\end{flalign*}
Thus, satisfying the claim that $a + c \equiv b+d \;(\bmod\; n)$.
\end{proof}

\subsection*{1.13 Theorem} 
\quad \textit{Let a, b, c, d and n be integers with $n > 0$. If $a \equiv b \;(\bmod\; n)$ and $c \equiv d \;(\bmod\; n)$ then $a - c \equiv b-d \;(\bmod\; n)$}

\begin{proof}
By definition, $a \equiv b \;(\bmod\; n)$ can be represented as:
\begin{flalign*}
&& (a-b) &= nk && \text{for a given integer k} \\
\Rightarrow && 1) b &= a - nk
\end{flalign*}

Similarly, 
\begin{flalign*}
&& (c-d) &= ns && \text{for a given integer s} \\
\Rightarrow && 2) d &= c - ns
\end{flalign*}
Now subtracting 1) and 2),
\begin{flalign*}
&& b-d &= (a - nk) - (c - ns) && \\
\Rightarrow && b-d &= a - c - n(k + s) && \\
\Rightarrow && n(k + s)&= (a - c) - (b-d)
\end{flalign*}
Thus, satisfying the claim that $a - c \equiv b-d \;(\bmod\; n)$.
\end{proof}

\subsection*{1.14 Theorem} 
\quad \textit{Let a, b, c, d and n be integers with $n > 0$. If $a \equiv b \;(\bmod\; n)$ and $c \equiv d \;(\bmod\; n)$ then $ac \equiv bd \;(\bmod\; n)$}

\begin{proof}
By definition, $a \equiv b \;(\bmod\; n)$ can be represented as:
\begin{flalign*}
&& (a-b) &= nk && \text{for a given integer k} \\
\Rightarrow && 1) b &= a - nk
\end{flalign*}

Similarly, 
\begin{flalign*}
&& (c-d) &= ns && \text{for a given integer s} \\
\Rightarrow && 2) d &= c - ns
\end{flalign*}
Now multiplying 1) and 2),
\begin{flalign*}
&& bd &= (a - nk)(c - ns) && \\
\Rightarrow && bd &= ac - a(ns)-c(nk) + (nk)(ns) && \\
\Rightarrow && bd &= ac - n*(a(s)-c(k) + n(k)(s))
\end{flalign*}
Thus, satisfying the claim that $ac \equiv bd \;(\bmod\; n)$.
\end{proof}

\subsection*{1.15 Theorem} 
\quad \textit{Let a, b and n be integers with $n > 0$. Show that if $a \equiv b \;(\bmod\; n)$ then $a^2 \equiv b^2 \;(\bmod\; n)$}

\begin{proof}
$a^2 \equiv b^2 \;(\bmod\; n)$ can be represented as $a(a) \equiv b(b) \;(\bmod\; n)$ by exponential property. Based off Theorem 1.14, we know that $ac \equiv bd \;(\bmod\; n)$ if $a \equiv b \;(\bmod\; n)$ and $c \equiv d \;(\bmod\; n)$. This shows that $a^2 \equiv b^2 \;(\bmod\; n)$ is true.
\end{proof}

\subsection*{1.16 Theorem} 
\quad \textit{Let a, b and n be integers with $n > 0$. Show that if $a \equiv b \;(\bmod\; n)$ then $a^3 \equiv b^3 \;(\bmod\; n)$}

\begin{proof}
By properties of exponents, $a^3 \equiv b^3 \;(\bmod\; n)$ can be represented as $(a)a^2 \equiv (a)b^2 \;(\bmod\; n)$. Using theorems 1.14 and 1.15, we can satisfy that $a^3 \equiv b^3 \;(\bmod\; n)$ is true.
\end{proof}

\subsection*{1.17 Theorem} 
\quad \textit{Let a, b, k and n be integers with $n > 0$ and $k > 1$. Show that if $a \equiv b \;(\bmod\; n)$ and $a^{k-1} \equiv b^{k-1} \;(\bmod\; n)$, then}
\begin{center}
    $a^{k} \equiv b^{k} \;(\bmod\; n)$
\end{center}

\begin{proof}
By properties of exponents, we can present the above statement as
    \begin{flalign*}
        && a^{k} &\equiv b^{k} \;(\bmod\; n) &&\\
        && a^{k}a^1a^{-1} &\equiv b^{k}b^1b^{-1} \;(\bmod\; n)&&\\
        && a^1a^{k-1} &\equiv b^1b^{k-1} \;(\bmod\; n)
    \end{flalign*}
Knowing that $a \equiv b \;(\bmod\; n)$ and $a^{k-1} \equiv b^{k-1} \;(\bmod\; n)$ and theorem 1.14, we can satisfy that $a^{k} \equiv b^{k} \;(\bmod\; n)$.
\end{proof}

\subsection*{1.18 Theorem} 
\quad \textit{Let a, b, k and n be integers with $n > 0$ and $k > 1$. Show that if $a \equiv b \;(\bmod\; n)$, then}

\begin{proof}
\textbf{Base case (k = 1):  }
    \begin{flalign*}
        && a^{k} &\equiv b^{k} \;(\bmod\; n) &&\\
        && a^{1} &\equiv b^{1} \;(\bmod\; n)
    \end{flalign*}
    Thus making the statement true if k = 1.\\
\textbf{Inductive Hypothesis: } Assume k = h + 1\\
\textbf{Inductive Step: }     
    \begin{flalign*}
        && a^{k} &\equiv b^{k} \;(\bmod\; n) &&\\
        && a^{h+1} &\equiv b^{h+1} \;(\bmod\; n) &&\\
        && a^{h}a &\equiv b^{h}b \;(\bmod\; n) && \text{(Exponential Property)}\\
    \end{flalign*}
    By theorem 1.14, we know that $a \equiv b \;(\bmod\; n)$ and $c \equiv d \;(\bmod\; n)$ then $ac \equiv bd \;(\bmod\; n)$. This helps satisfies $a^{h+1} \equiv b^{h+1} \;(\bmod\; n)$ since $a^{h} \equiv b^{h} \;(\bmod\; n)$ and $a \equiv b \;(\bmod\; n)$.
\end{proof}

\subsection*{1.19 Theorem}
\begin{itemize}
    \item \textbf{1.12 Theorem} $n = 3, a = 2, b = 17, c = 1$ and $d = 19$ then $2 + 1 \equiv 17+19 \;(\bmod\; 3)$
    \item \textbf{1.13 Theorem} $n = 3, a = 2, b = 17, c = 1$ and $d = 19$ then $2 - 1 \equiv 17-19 \;(\bmod\; 3)$
    \item \textbf{1.14 Theorem} $n = 3, a = 2, b = 17, c = 1$ and $d = 19$ then $2 *1 \equiv 17*19 \;(\bmod\; 3)$
    \item \textbf{1.15 Theorem} $n = 3, a = 2, b = 17$ then $2^2 \equiv 17^2 \;(\bmod\; 3)$
    \item \textbf{1.16 Theorem} $n = 3, a = 2, b = 17$ then $2^3 \equiv 17^3 \;(\bmod\; 3)$
    \item \textbf{1.17 Theorem} $n = 3, a = 2, b = 17$ then $2^k \equiv 17^k \;(\bmod\; 3)$ where $k \in \mathbb{Z}$ and $k > 1$
    \item \textbf{1.18 Theorem} $n = 3, a = 2, b = 17$ then $2^k \equiv 17^k \;(\bmod\; 3)$ where $k \in \mathbb{Z}$ and $k > 1$
\end{itemize}

\subsection*{1.20 Theorem} 
\quad \textit{Let a, b, c and n be integers for which $ac \equiv bc \;(\bmod\; n)$. Can we conclude that $a \equiv b \;(\bmod\; n)$?}

\begin{proof}
By counterexample, given $a=1, b= 17, c=2$ and $n =3$ where $ac \equiv bc \;(\bmod\; n)$ with $1(2) \equiv 17(2) \;(\bmod\; 3)$. However, we can not conclude that $1 \equiv 17 \;(\bmod\; 3)$. 
\end{proof}

\subsection*{1.21 Theorem} 
\quad \textit{Let a natural number n be expressed in base 10 as}
\begin{center}
    $n = a_ka_{k-1}...a_1a_0$
\end{center}
\textit{If $m=a_k+a_{k+1}+...+a_1+a_0$, then $n \equiv m \;(\bmod\; 3)$.}

\begin{proof}
By definition, $n \equiv m \;(\bmod\; 3)$ can be expressed as:
    \begin{flalign*}
        && 3 &\vert n-m &&\\
    \end{flalign*}
\end{proof}
By theorem 1.2, for this theorem to be true, $3 \vert n$ and $3 \vert m$ must be true.

\subsection*{1.22 Theorem} 
\quad \textit{If a natural number is divisible by 3, then when expressed in base 10 , the sum of its digits is divisible by 3.}

\begin{proof}
Suppose natural number $n = a_ka_{k-1}...a_1a_0$ and sum of its digits $m=a_k+a_{k+1}+...+a_1+a_0$. We can use theorem 1.21 to prove this.
\end{proof}

\subsection*{1.23 Theorem} 
\quad \textit{If the sum of the digits of a natural number expressed in base 10 is divisible by 3, then the number is divisible by 3 as well.}

\begin{proof}
Suppose natural number $n = a_ka_{k-1}...a_1a_0$ and sum of its digits $m=a_k+a_{k+1}+...+a_1+a_0$. We can use theorem 1.21 to prove this.
\end{proof}

\subsection*{1.24 Exercise} 
\quad \textit{Suppose natural number $n = a_ka_{k-1}...a_1a_0$ and sum of its digits $m=a_k+a_{k+1}+...+a_1+a_0$. If the sum is divisible by 6, then the natural number is 3.}

\begin{proof}
By theorem 1.21, since 6 is divisible by 3 as well.
\end{proof}

\subsection*{1.25 Exercise} 

\begin{enumerate}
    \item $m = 25, n = 7$ 
    \begin{flalign*}
        m &= nq + r &&\\
        25 &= 7q + r &&\\
        25 &= 7\times3 + 4 &&\\
        3 &= 7\times1 + 3 &&\\
        7 &= 4\times1 + 3 &&\\
    \end{flalign*}
    $q = 3, r = 4$
    \item $m = 277, n = 4$ 
    \begin{flalign*}
        m &= nq + r &&\\
        277 &= 4q + r &&\\
        277 &= 4\times69 + 1 &&\\
        4 &= 4\times1 + 0
    \end{flalign*}
    $q = 69, r = 1$
    \item $m = 33, n = 11$ 
    \begin{flalign*}
        m &= nq + r &&\\
        33 &= 11q + r &&\\
        33 &= 11\times3 + 0
    \end{flalign*}
    $q = 3, r = 0$
    \item $m = 33, n = 45$ 
    \begin{flalign*}
        m &= nq + r &&\\
        33 &= 45q + r &&\\
        33 &= 45\times0 + 33
    \end{flalign*}
    $q = 1, r = -12$
    
\end{enumerate}

\subsection*{1.26 Theorem} 
\quad \textit{Prove the existence part of the Division Algorithm. (Hint: Given n and m, how will you define q? Once you choose this q, then how is r chosen? Then show that $0 \leq r \leq n -1$.)}

\begin{proof}

\end{proof}

\subsection*{1.27 Theorem} 
\quad \textit{Prove the uniqueness part of the Division Algorithm. (Hint: If $nq+r = nq^{'}+r^{'}$, then $nq-nq^{'}=r^{'}-r$. Use what you know about $r$ and $r^{'}$ as part of your argument that $q = q^{'}$.}

\begin{proof}
\end{proof}

\subsection*{1.28 Theorem} 
\quad \textit{Let a, b and n be integers with n > 0.  Then $a \equiv b \;(\bmod\; n)$ if and only if a and b have the same remainder when divided by n. Equivalently, $a \equiv b \;(\bmod\; n)$ if and only if when $a=nq_1+r_1$ $(0\leq r_1 \leq n-1)$ and $b = nq_2+r_2$ $ (0\leq r_2 \leq n-1)$, then $r_1=r_2$.}

\begin{proof}
\end{proof}

\subsection*{1.29 Question} 
\quad \textit{Do every two integers have a least one common divisor?}

\begin{proof}
\end{proof}

\subsection*{1.30 Question} 
\quad \textit{Can two integers have infinitely many common divisors?}

\begin{proof}
\end{proof}

\subsection*{1.31 Exercise} 
\quad \textit{Find the the following greatest common divisors. Which pairs are relatively prime?}

\begin{enumerate}
    \item (36, 22)
    \item (45, -15)
    \item (-296, -88)
    \item (0, 256)
    \item (15, 28)
    \item (1, -2436)
\end{enumerate}

\subsection*{1.32 Theorem} 
\quad \textit{Let a, n, b, r and k be integers. If $a = nb +r$ and $k \vert a$ and $k \vert b$, then $k \vert r$.}

\begin{proof}
\end{proof}

\subsection*{1.33 Theorem} 
\quad \textit{Let a, b, $n_1$, and $r_1$ be integers with a and b not both $0$. If $a = n_{1}b+r_1$, then $(a,b)=(b,r_1)$.}

\begin{proof}
\end{proof}

\subsection*{1.34 Exercise} 
\quad \textit{Use the above theorem (Euclidean Algorithm) to show that if $a=51$ and $b=15$, then $(51, 15)=(6,3)=3$.}

\begin{proof}
\end{proof}

\subsection*{1.35 Exercise (Euclidean Algorithm)} 
\quad \textit{Devise a procedure for finding the greatest common divisor of two integers using the previous theorem and the Division Algorithm.}

\begin{proof}
\end{proof}

\subsection*{1.36 Exercise} 
\quad \textit{Use the Euclidean Algorithm to find}

\begin{enumerate}
    \item (96, 112)
    \item (162, 31)
    \item (0, 256)
    \item (-288, -166)
    \item (1, -2436)
\end{enumerate}

\subsection*{1.37 Exercise} 
\quad \textit{Find integers x and y such that $162x+31y=1$.}

\begin{proof}
\end{proof}

\subsection*{1.38 Exercise} 
\quad \textit{Let ?}

\begin{proof}
\end{proof}

\subsection*{1.39 Exercise} 
\quad \textit{Let ?}

\begin{proof}
\end{proof}

\subsection*{1.40 Exercise} 
\quad \textit{Let ?}

\begin{proof}
\end{proof}

\subsection*{1.41 Exercise} 
\quad \textit{Let ?}

\begin{proof}
\end{proof}

\subsection*{1.42 Exercise} 
\quad \textit{Let ?}

\begin{proof}
\end{proof}

\subsection*{1.43 Exercise} 
\quad \textit{Let ?}

\begin{proof}
\end{proof}

\subsection*{1.44 Exercise} 
\quad \textit{Let ?}

\begin{proof}
\end{proof}

\subsection*{1.45 Exercise} 
\quad \textit{Let ?}

\begin{proof}
\end{proof}

\subsection*{1.46 Exercise} 
\quad \textit{Let ?}

\begin{proof}
\end{proof}

\subsection*{1.47 Exercise} 
\quad \textit{Let ?}

\begin{proof}
\end{proof}

\subsection*{1.48 Exercise} 
\quad \textit{Let ?}

\begin{proof}
\end{proof}

\subsection*{1.49 Exercise} 
\quad \textit{Let ?}

\begin{proof}
\end{proof}

\subsection*{1.50 Exercise} 
\quad \textit{Let ?}

\begin{proof}
\end{proof}

\subsection*{1.51 Exercise} 
\quad \textit{Let ?}

\begin{proof}
\end{proof}

\subsection*{1.52 Exercise} 
\quad \textit{Let ?}

\begin{proof}
\end{proof}

\subsection*{1.53 Exercise} 
\quad \textit{Let ?}

\begin{proof}
\end{proof}

\subsection*{1.54 Exercise} 
\quad \textit{Let ?}

\begin{proof}
\end{proof}

\subsection*{1.55 Exercise} 
\quad \textit{Let ?}

\begin{proof}
\end{proof}

\subsection*{1.56 Exercise} 
\quad \textit{Let ?}

\begin{proof}
\end{proof}

\subsection*{1.57 Exercise} 
\quad \textit{Let ?}

\begin{proof}
\end{proof}

\subsection*{1.58 Corollary} 
\quad \textit{Let ?}

\begin{proof}
\end{proof}

\subsection*{1.59 Exercise} 
\quad \textit{Let ?}

\begin{proof}
\end{proof}


\end{document}