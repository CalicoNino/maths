\documentclass{article}
\usepackage[utf8]{inputenc}
\usepackage[english]{babel}

\usepackage{amsthm}

\title{Divide \& Conquer Proofs}
\author{Karim El Shenawy}
\date{December 2020}

\usepackage{natbib}
\usepackage{graphicx}

\begin{document}

\maketitle

\section*{Introduction}
This course notebook is the collection of theorem proofs, exercises and answers from Unit 1 of the Number Theory Through Inquiry (Mathematical Association of America Textbooks).

\section*{Divisibility \& Congruence}

\subsection*{1.1 Theorem} 
\quad \textit{Let a, b, and c be integers. If $a \vert b$ and $a \vert c$, then $a \vert (b+c)$.}

\begin{proof}
Our hypothesises that $a \vert b$ and $a \vert c$ both mean respectively, by definition, that $b = k_1a$ and $c = k_2a$ for some integer $k_1$ and $k_2$. Also by definition, $a \vert (b+c)$ means that $b + c = k_3a$ for some integer $k_3$. Using that we can say that $b + c = k_1a + k_2a = a(k_1+k_2) = k_3a$. Thus by definition, $a \vert a(k_1+k_2)$, knowing that $(b+c) = a(k_1+k_2)$, we can satisfy the definition of $a \vert (b+c)$.
\end{proof}


\subsection*{1.2 Theorem} 
\quad \textit{Let a, b, and c be integers. If $a \vert b$ and $a \vert c$, then $a \vert (b-c)$.}

\begin{proof}
Our hypothesises that $a \vert b$ and $a \vert c$ both mean respectively, by definition, that $b = k_1a$ and $c = k_2a$ for some integer $k_1$ and $k_2$. Also by definition, $a \vert (b-c)$ means that $b - c = k_3a$ for some integer $k_3$. Using that we can say that $b - c = k_1a - k_2a = a(k_1 - k_2) = k_3a$. Thus by definition, $a \vert a(k_1 - k_2)$, knowing that $(b - c) = a(k_1 - k_2)$, we can satisfy the definition of $a \vert (b - c)$.
\end{proof}

\subsection*{1.3 Theorem} 
\quad \textit{Let a, b, and c be integers. If $a \vert b$ and $a \vert c$, then $a \vert bc$.}

\begin{proof}
Our hypothesises that $a \vert b$ and $a \vert c$ both mean respectively, by definition, that $b = k_1a$ and $c = k_2a$ for some integer $k_1$ and $k_2$. Also by definition, $a \vert bc$ means that $bc = (k_1a)(k_2a) = k_1k_2a^2$. Using that we can say that $bc = k_1k_2a^2 = (k_1k_2a)a$. Thus by definition, $a \vert (k_1k_2a)$, knowing that $bc= (k_1k_2a)a$, we can satisfy the definition of $a \vert bc$.
\end{proof}

\subsection*{1.4 Question} 
\quad \textit{Can you weaken the hypothesis of the previous theorem and still prove the conclusion? Can you keep the same hypothesis, but replace the conclusion by the stronger conclusion that $a^2 \vert bc$ and still prove the theorem?}

We can say that 

\subsection*{1.5 Question} 
\quad \textit{Can you formulate your own conjecture along the lines of the above theorems and then prove it to make your theorem?}

\subsection*{1.6 Theorem} 
\quad \textit{Let a, b, and c be integers. If $a \vert b$, then $a \vert bc$.}

\begin{proof}
Based off of Theorem 1.4, we can deduce that 
\end{proof}

\subsection*{1.7 Exercise} 
\begin{enumerate}
    \item \textit{Is $45 \equiv 9 \;(\bmod\; 4)$?}
    \newline Yes, since $4 \vert (45-9) = 4 \vert 36$ and 4 does divide 36.
    \item \textit{Is $37 \equiv 2 \;(\bmod\; 5)$?}
    \newline Yes, since $5 \vert (37-2) = 5 \vert 35$ and 5 does divide 35.
    \item \textit{Is $37 \equiv 3 \;(\bmod\; 5)$?}
    \newline No, since $5 \vert (37-3) = 5 \vert 34$ and 5 does not divide 34.
    \item \textit{Is $37 \equiv -3 \;(\bmod\; 5)$?}
    \newline Yes, since $5 \vert (37-(-3)) = 5 \vert 40$ and 5 does divide 40.
\end{enumerate}

\subsection*{1.8 Exercise} 
\begin{enumerate}
    \item $m \equiv 0 \;(\bmod\; 3)$.
    \newline $m$ can be any integer such that $3 \vert m$ thus m can be any integer from \{$-3(N), -3(N-1)...-3(1), 3(1),3(2),3(3),12,15...3(N-1), 3(N)$\} where is $N$ is the length of the set.
    \item $m \equiv 1 \;(\bmod\; 3)$.
    \newline $m$ can be any integer such that $3 \vert (m+1)$ thus m can be any integer from \{$-3(N)-1, -3(N-1)-1...-3(1)-1, 3(1)-1,3(2)-1,3(3)-1,11,14...3(N-1)-1, 3(N)-1$\} where is $N$ is the length of the set.
    \item $m \equiv 2 \;(\bmod\; 3)$.
    \newline $m$ can be any integer such that $3 \vert (m+2)$ thus m can be any integer from \{$-3(N)-2, -3(N-1)-2...-3(1)-2, 3(1)-2,3(2)-2,3(3)-2,10,13...3(N-1)-2, 3(N)-2$\} where is $N$ is the length of the set.
    \item $m \equiv 3 \;(\bmod\; 3)$.
    \newline $m$ can be any integer such that $3 \vert m$ thus m can be any integer from \{$-3(N), -3(N-1)...-3(1), 3(1),3(2),3(3),12,15...3(N-1), 3(N)$\} where is $N$ is the length of the set.
    \item $m \equiv 4 \;(\bmod\; 3)$.
    \newline $m$ can be any integer such that $3 \vert m+4$ thus m can be any integer from \{$-3(N)-1, -3(N-1)-1...-3(1)-1, 3(1)-1,3(2)-1,3(3)-1,11,14...3(N-1)-1, 3(N)-1$\} where is $N$ is the length of the set.
\end{enumerate}

\subsection*{1.9 Theorem} 
\quad \textit{Let a, and n be integers with $n > 0$. Then $a \equiv a \;(\bmod\; n)$.}

\begin{proof}
\end{proof}

\subsection*{1.10 Theorem} 
\quad \textit{Let a, b, and n be integers with $n > 0. Then a \equiv a \;(\bmod\; n)$.}

\begin{proof}
\end{proof}

\end{document}
