\documentclass{article}
\usepackage[utf8]{inputenc}
\usepackage[english]{babel}

\usepackage{amsthm}
\usepackage{amsmath}
\usepackage{amssymb}
\usepackage{fullpage}
\usepackage[round]{natbib}
\usepackage{multirow}
\usepackage{booktabs}
\usepackage{tabularx}
\usepackage{graphicx}
\usepackage{float}
\usepackage{hyperref}

\usepackage[round]{natbib}

\title{Introduction To\\Mathematical Philospohy\\Summary}
\author{Karim El Shenawy}
\date{December 2021}

\usepackage{natbib}
\usepackage{graphicx}
\setlength{\parindent}{2em}
\setlength{\parskip}{1em}
\renewcommand{\baselinestretch}{1.5}


\begin{document}

\maketitle
\thispagestyle{empty}
\newpage
\pagenumbering{arabic}
\tableofcontents

\newpage

\section{Introduction}

This document pretains the summary of \textit{Introduction to Mathematical Philosophy} by \textbf{Bertrand Russel}. Important Theories and Highlighted Statements will be included in each section as apart of their respective chapter. Each chapter will also include supplementary research made to add more information.

\section{The Series of Natural Numbers}

\section{Defintion of Number}

\section{Finitude and Mathematical Induction}

\section{The Defintion of Order}

\section{Kinds of Relations}

\section{Similarity of Relations}

\section{Rational, Real, and Complex Numbers}

\section{Infinite Cardinal Numbers}

\section{Infinite Series Ordinals}

\section{Limits and Continuity}

\section{Limits and Continuity of Functions}

\section{Selections and the Multiplicative Axiom}

\section{The Axiom of Infinity and Logical Types}

\section{Incompatibility and the Theory of Deduction}

\section{Propositional Functions}

\section{Descriptions}

\section{Classes}

\section{Mathematics and Logic}




\end{document}